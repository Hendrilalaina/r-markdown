% Options for packages loaded elsewhere
\PassOptionsToPackage{unicode}{hyperref}
\PassOptionsToPackage{hyphens}{url}
%
\documentclass[
]{article}
\usepackage{amsmath,amssymb}
\usepackage{iftex}
\ifPDFTeX
  \usepackage[T1]{fontenc}
  \usepackage[utf8]{inputenc}
  \usepackage{textcomp} % provide euro and other symbols
\else % if luatex or xetex
  \usepackage{unicode-math} % this also loads fontspec
  \defaultfontfeatures{Scale=MatchLowercase}
  \defaultfontfeatures[\rmfamily]{Ligatures=TeX,Scale=1}
\fi
\usepackage{lmodern}
\ifPDFTeX\else
  % xetex/luatex font selection
\fi
% Use upquote if available, for straight quotes in verbatim environments
\IfFileExists{upquote.sty}{\usepackage{upquote}}{}
\IfFileExists{microtype.sty}{% use microtype if available
  \usepackage[]{microtype}
  \UseMicrotypeSet[protrusion]{basicmath} % disable protrusion for tt fonts
}{}
\makeatletter
\@ifundefined{KOMAClassName}{% if non-KOMA class
  \IfFileExists{parskip.sty}{%
    \usepackage{parskip}
  }{% else
    \setlength{\parindent}{0pt}
    \setlength{\parskip}{6pt plus 2pt minus 1pt}}
}{% if KOMA class
  \KOMAoptions{parskip=half}}
\makeatother
\usepackage{xcolor}
\usepackage[margin=1in]{geometry}
\usepackage{color}
\usepackage{fancyvrb}
\newcommand{\VerbBar}{|}
\newcommand{\VERB}{\Verb[commandchars=\\\{\}]}
\DefineVerbatimEnvironment{Highlighting}{Verbatim}{commandchars=\\\{\}}
% Add ',fontsize=\small' for more characters per line
\usepackage{framed}
\definecolor{shadecolor}{RGB}{248,248,248}
\newenvironment{Shaded}{\begin{snugshade}}{\end{snugshade}}
\newcommand{\AlertTok}[1]{\textcolor[rgb]{0.94,0.16,0.16}{#1}}
\newcommand{\AnnotationTok}[1]{\textcolor[rgb]{0.56,0.35,0.01}{\textbf{\textit{#1}}}}
\newcommand{\AttributeTok}[1]{\textcolor[rgb]{0.13,0.29,0.53}{#1}}
\newcommand{\BaseNTok}[1]{\textcolor[rgb]{0.00,0.00,0.81}{#1}}
\newcommand{\BuiltInTok}[1]{#1}
\newcommand{\CharTok}[1]{\textcolor[rgb]{0.31,0.60,0.02}{#1}}
\newcommand{\CommentTok}[1]{\textcolor[rgb]{0.56,0.35,0.01}{\textit{#1}}}
\newcommand{\CommentVarTok}[1]{\textcolor[rgb]{0.56,0.35,0.01}{\textbf{\textit{#1}}}}
\newcommand{\ConstantTok}[1]{\textcolor[rgb]{0.56,0.35,0.01}{#1}}
\newcommand{\ControlFlowTok}[1]{\textcolor[rgb]{0.13,0.29,0.53}{\textbf{#1}}}
\newcommand{\DataTypeTok}[1]{\textcolor[rgb]{0.13,0.29,0.53}{#1}}
\newcommand{\DecValTok}[1]{\textcolor[rgb]{0.00,0.00,0.81}{#1}}
\newcommand{\DocumentationTok}[1]{\textcolor[rgb]{0.56,0.35,0.01}{\textbf{\textit{#1}}}}
\newcommand{\ErrorTok}[1]{\textcolor[rgb]{0.64,0.00,0.00}{\textbf{#1}}}
\newcommand{\ExtensionTok}[1]{#1}
\newcommand{\FloatTok}[1]{\textcolor[rgb]{0.00,0.00,0.81}{#1}}
\newcommand{\FunctionTok}[1]{\textcolor[rgb]{0.13,0.29,0.53}{\textbf{#1}}}
\newcommand{\ImportTok}[1]{#1}
\newcommand{\InformationTok}[1]{\textcolor[rgb]{0.56,0.35,0.01}{\textbf{\textit{#1}}}}
\newcommand{\KeywordTok}[1]{\textcolor[rgb]{0.13,0.29,0.53}{\textbf{#1}}}
\newcommand{\NormalTok}[1]{#1}
\newcommand{\OperatorTok}[1]{\textcolor[rgb]{0.81,0.36,0.00}{\textbf{#1}}}
\newcommand{\OtherTok}[1]{\textcolor[rgb]{0.56,0.35,0.01}{#1}}
\newcommand{\PreprocessorTok}[1]{\textcolor[rgb]{0.56,0.35,0.01}{\textit{#1}}}
\newcommand{\RegionMarkerTok}[1]{#1}
\newcommand{\SpecialCharTok}[1]{\textcolor[rgb]{0.81,0.36,0.00}{\textbf{#1}}}
\newcommand{\SpecialStringTok}[1]{\textcolor[rgb]{0.31,0.60,0.02}{#1}}
\newcommand{\StringTok}[1]{\textcolor[rgb]{0.31,0.60,0.02}{#1}}
\newcommand{\VariableTok}[1]{\textcolor[rgb]{0.00,0.00,0.00}{#1}}
\newcommand{\VerbatimStringTok}[1]{\textcolor[rgb]{0.31,0.60,0.02}{#1}}
\newcommand{\WarningTok}[1]{\textcolor[rgb]{0.56,0.35,0.01}{\textbf{\textit{#1}}}}
\usepackage{graphicx}
\makeatletter
\def\maxwidth{\ifdim\Gin@nat@width>\linewidth\linewidth\else\Gin@nat@width\fi}
\def\maxheight{\ifdim\Gin@nat@height>\textheight\textheight\else\Gin@nat@height\fi}
\makeatother
% Scale images if necessary, so that they will not overflow the page
% margins by default, and it is still possible to overwrite the defaults
% using explicit options in \includegraphics[width, height, ...]{}
\setkeys{Gin}{width=\maxwidth,height=\maxheight,keepaspectratio}
% Set default figure placement to htbp
\makeatletter
\def\fps@figure{htbp}
\makeatother
\setlength{\emergencystretch}{3em} % prevent overfull lines
\providecommand{\tightlist}{%
  \setlength{\itemsep}{0pt}\setlength{\parskip}{0pt}}
\setcounter{secnumdepth}{-\maxdimen} % remove section numbering
\ifLuaTeX
  \usepackage{selnolig}  % disable illegal ligatures
\fi
\usepackage{bookmark}
\IfFileExists{xurl.sty}{\usepackage{xurl}}{} % add URL line breaks if available
\urlstyle{same}
\hypersetup{
  pdftitle={Import/Export},
  pdfauthor={Hendrilalaina},
  hidelinks,
  pdfcreator={LaTeX via pandoc}}

\title{Import/Export}
\author{Hendrilalaina}
\date{2025-03-17}

\begin{document}
\maketitle

\section{Import}\label{import}

\subsection{CSV}\label{csv}

This is the import of ``Affairs.csv''

\begin{Shaded}
\begin{Highlighting}[]
\FunctionTok{library}\NormalTok{(readr)}
\NormalTok{affairs\_csv }\OtherTok{\textless{}{-}} \FunctionTok{read\_csv}\NormalTok{(}\AttributeTok{file =} \StringTok{"./../RUltimateMaterial{-}main/data/Affairs.csv"}\NormalTok{)}
\end{Highlighting}
\end{Shaded}

\begin{verbatim}
## Rows: 601 Columns: 9
## -- Column specification ----------------------------
## Delimiter: ","
## chr (2): sex, child
## dbl (7): age, ym, religious, education, occupation, rate, nbaffairs
## 
## i Use `spec()` to retrieve the full column specification for this data.
## i Specify the column types or set `show_col_types = FALSE` to quiet this message.
\end{verbatim}

\begin{Shaded}
\begin{Highlighting}[]
\FunctionTok{View}\NormalTok{(affairs\_csv)}
\end{Highlighting}
\end{Shaded}

\subsection{Excel}\label{excel}

This is the import of ``Affaires.xlsx''

\begin{Shaded}
\begin{Highlighting}[]
\FunctionTok{library}\NormalTok{(readxl)}
\NormalTok{affairs\_xlsx }\OtherTok{\textless{}{-}} \FunctionTok{read\_excel}\NormalTok{(}\StringTok{"./../RUltimateMaterial{-}main/data/Affairs.xlsx"}\NormalTok{)}
\FunctionTok{View}\NormalTok{(affairs\_xlsx)}
\end{Highlighting}
\end{Shaded}

\subsection{JSON}\label{json}

JSON is a JavaScript Object Notation

\begin{Shaded}
\begin{Highlighting}[]
\FunctionTok{library}\NormalTok{(jsonlite)}
\NormalTok{affairs\_json }\OtherTok{\textless{}{-}} \FunctionTok{read\_json}\NormalTok{(}\AttributeTok{path =} \StringTok{"./../RUltimateMaterial{-}main/data/Affairs.JSON"}\NormalTok{, }\AttributeTok{simplifyVector =}\NormalTok{ T)}
\FunctionTok{View}\NormalTok{(affairs\_json)}
\end{Highlighting}
\end{Shaded}

\subsection{SPSS}\label{spss}

SPSS is a statistical program with its own file format.

\begin{Shaded}
\begin{Highlighting}[]
\FunctionTok{library}\NormalTok{(foreign)}
\NormalTok{affairs\_spss }\OtherTok{\textless{}{-}} \FunctionTok{read.table}\NormalTok{(}\StringTok{"./../RUltimateMaterial{-}main/data/Affairs.sps"}\NormalTok{)}
\FunctionTok{View}\NormalTok{(affairs\_spss)}
\end{Highlighting}
\end{Shaded}

\subsection{RDA}\label{rda}

RDA is a native R format, which provides compression. RDA saves the
object names and can store more than just one object.

\begin{Shaded}
\begin{Highlighting}[]
\FunctionTok{load}\NormalTok{(}\AttributeTok{file =} \StringTok{"./../RUltimateMaterial{-}main/data/Affairs.RDA"}\NormalTok{)}
\end{Highlighting}
\end{Shaded}

\subsection{RDS}\label{rds}

RDS is the new native R format. RDS does not store the object name and
can only contain one object. It has a slightly better compression rate
compared to RDA.

\begin{Shaded}
\begin{Highlighting}[]
\NormalTok{affairs\_rds }\OtherTok{\textless{}{-}} \FunctionTok{readRDS}\NormalTok{(}\StringTok{"./../RUltimateMaterial{-}main/data/Affairs.RDS"}\NormalTok{)}
\FunctionTok{View}\NormalTok{(affairs\_rds)}
\end{Highlighting}
\end{Shaded}

\section{Export}\label{export}

\subsection{CSV}\label{csv-1}

Export of ``affairs\_csv''

\begin{Shaded}
\begin{Highlighting}[]
\FunctionTok{write.csv}\NormalTok{(}\AttributeTok{x =}\NormalTok{ affairs\_csv, }\AttributeTok{file =} \StringTok{"./export/Affairs.csv"}\NormalTok{)}
\end{Highlighting}
\end{Shaded}

\subsection{JSON}\label{json-1}

Export of ``affairs\_json''

\begin{Shaded}
\begin{Highlighting}[]
\FunctionTok{write\_json}\NormalTok{(}\AttributeTok{x =}\NormalTok{ affairs\_json, }\AttributeTok{path =} \StringTok{"./export/Affairs.json"}\NormalTok{)}
\end{Highlighting}
\end{Shaded}

\subsection{SPSS}\label{spss-1}

Export of ``affairs\_spss''

\begin{Shaded}
\begin{Highlighting}[]
\FunctionTok{write.foreign}\NormalTok{(affairs\_spss, }\AttributeTok{datafile =} \StringTok{"./../RUltimateMaterial{-}main/data/Affairs.sps"}\NormalTok{, }\AttributeTok{codefile =} \StringTok{"./export/Affairs.code"}\NormalTok{, }\AttributeTok{package =} \StringTok{"SPSS"}\NormalTok{)}
\end{Highlighting}
\end{Shaded}

\subsection{RDA and RDS}\label{rda-and-rds}

RDA and RDS are native R formats, which store data compressed.

\begin{Shaded}
\begin{Highlighting}[]
\FunctionTok{save}\NormalTok{(affairs\_csv, }\AttributeTok{file =} \StringTok{"./export/Affairs.RDA"}\NormalTok{)}
\FunctionTok{saveRDS}\NormalTok{(affairs\_csv, }\AttributeTok{file =} \StringTok{"./export/Affairs.RDS"}\NormalTok{)}
\end{Highlighting}
\end{Shaded}


\end{document}
